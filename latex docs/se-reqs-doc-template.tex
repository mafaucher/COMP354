\documentclass[12pt]{article}

\pagestyle{empty}
\setcounter{secnumdepth}{2}
\newcommand{\systemName}{text to insert}
\topmargin=0cm
\oddsidemargin=0cm
\textheight=22.0cm
\textwidth=16cm
\parindent=0cm
\parskip=0.15cm
\topskip=0truecm
\raggedbottom
\abovedisplayskip=3mm
\belowdisplayskip=3mm
\abovedisplayshortskip=0mm
\belowdisplayshortskip=2mm
\normalbaselineskip=12pt
\normalbaselines

\begin{document}

\vspace*{0.5in}
\centerline{\bf\Large Requirements Document}

\vspace*{0.5in}
\centerline{\bf\Large Team 1}

\vspace*{0.5in}
\centerline{\bf\Large 17 January 2012}

\vspace*{1.5in}
\begin{table}[htbp]
\caption{Team}
\begin{center}
\begin{tabular}{|r | c|}
\hline
{\bf Name} & {\bf ID Number} \\
Jeffrey How & 9430954 \\
William Ling & 9193480 \\
Thomas Paulin & 9333630 \\
Kai Wang & 5652723 \\
Jonathan Bergeron & 9764453 \\
\hline
\end{tabular}
\end{center}
\end{table}

\clearpage

\section{System}

\subsection{Purpose}
The purpose of this document is to define requirements for system \systemName.
\newline\newline
The intended audience of this document is described in table 2
\newline
\begin{table}[htbp]
\caption{Targeted audience of this document.}
\begin{center}
\begin{tabular}{|l | l|}
\hline
{\bf Group of the readers} & {\bf Reasons for reading}\\ \cline{1-2}
Users and customers & To give feedback about the requirements\\ \cline{1-2}
System developers & To understand what functions and properties the system must contain\\ \cline{1-2}
Testers & To test the system against the requirements\\ \cline{1-2}
Project team & To follow-up the status of the project against the requirements\\ \cline{1-2}
\hline
\end{tabular}
\end{center}
\end{table}

\subsection{Context}
Our goal is to develop a task and time management software system for game development projects. 

Users of this system include programmers, graphic artists, model artists, webmasters, project managers and designers.

Our software creates and organizes a list of tasks on per user basis along with the time requirements so they can properly allocate resources needed to achieve each task.
\subsection{Business Goals}
Our goal is to offer game development companies a system to properly manage tasks and time related to a project.

Many methods of a various degree of efficiency are currently used to manage tasks and time. Some game developers use spreadsheets or paper solutions to name a few in order to manage resources.

Our solution to this problem is to offer them an easy to use system that standardizes the development process and helps coordination of tasks in order to achieve a comon goal for the team.
\section{Domain Concepts}


This chapter gives a brief introduction to the problem domain. It might include
    Textual definitions of the most important domain concepts (jargon)

\section{Actors}
Programmers
Artists
Webmaster


\section{Use Cases}

\subsection{System overview}
This section is a high level description of the intended solution (=the system). It might include

    a list of essential features of the system
    a graph (for example a use case diagram) that defines the users and the main functions of the system

\begin{figure}[htbp]
%insert diagram here
\caption{Use Case Diagram}
\label{fig:use-case-diagram}
\end{figure}

\subsubsection{Use Case 1} \label{uc:1}

\noindent
{\bf Name}\\
Give a name.

\noindent
{\bf Summary}\\
A short summary/description/story.

\noindent
{\bf Actors}\\

\noindent
{\bf Precondition}\\

\noindent
{\bf Main Scenario}\\
\vspace*{-0.2in}
\begin{enumerate}
\item Describe step 1.
\item Describe step 2.
\item Describe step 3.
\end{enumerate}

\noindent
{\bf Exceptions}\\

\noindent
{\bf Postcondition}\\

\noindent
{\bf Priority}\\

\noindent
{\bf Traces to Test Cases}\\
Add when test cases done.

\subsubsection{Use Case 2} \label{uc:2}

\section{Functional requirements}

\section{Non-Functional Constraints}

\section{Constraints}


\section{Solution ideas}

\section{Data Dictionary}

\section{References}

\appendix

\section{Description of File Format: Tasks}

Describe input file format.

\section{Description of File Format: Persons}

Describe output file format.

\end{document}
